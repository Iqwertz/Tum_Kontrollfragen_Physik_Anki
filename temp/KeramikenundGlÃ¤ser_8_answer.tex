\documentclass[11pt,a4paper]{article}
\usepackage[utf8]{inputenc}
\usepackage[german]{babel}
\usepackage{amsmath}
\usepackage{amsfonts}
\usepackage{subfig}
\usepackage{amssymb}
\usepackage{siunitx,physics}
\usepackage{mathtools}
\usepackage{graphicx}
%\usepackage{Here}
\usepackage[version=4]{mhchem}
\usepackage{url}
\usepackage{setspace}
\usepackage[left=2.5cm,right=2.5cm,top=2.5cm,bottom=2cm]{geometry}
[biblography=totocnumbered]
\usepackage{fancyhdr}
\usepackage{scrextend}
\usepackage{hyperref}
\pagenumbering{gobble}

\makeatletter
\newcommand\bigcdot{\mathpalette\bigcdot@{.5}}
\newcommand\bigcdot@[2]{\mathbin{\vcenter{\hbox{\scalebox{#2}{$\m@th#1\bullet$}}}}}
\makeatother

\makeatletter
%\renewcommand*\bib@heading{%
%  \subsection*{}%
%  \@mkboth{\refname}{\refname}}
%\makeatother
\numberwithin{equation}{section}
\numberwithin{figure}{section}

\renewcommand{\labelitemii}{\labelitemfont$\vartriangleright$}
\begin{document}\\
\begin{addmargin}[25pt]{0pt}
Glaskeramik besteht aus Kristalliten, welche in eine glasförmige Matrix eingebettet sind. Glaskeramiken bestehen hauptsächlich aus \ce{SiO2}, außerdem sind \ce{Na2O}, \ce{Al2O3}, \ce{B2O3}, \ce{TiO2} und \ce{As2O3} in ihr enthalten. Die Kristallite und das Glas haben einen ähnlichen Brechungsindex wodurch die Glaskeramik durchsichtig wird. Glaskeramiken haben einen niedrigen Wärmeausdehnungskoeffizienten, daher sind sie ideal zum Bau von Kochfeldern oder Haltern in Hochtemperatur-Brennstoffzellen. Zur Herstellung von Glaskeramiken werden zuerst die Bestandteile geschmolzen, danach bei einer etwas niedrigeren Temperatur geformt, als nächstes wird die Schmelze abgekühlt und danach bei verschiedenen Temperaturen gehalten damit erst Keime sich bilden können und diese danach wachsen. Schließlich ist die Glaskeramik in der gewünschten Form erstarrt. \\
\end{addmargin}


\end{document}