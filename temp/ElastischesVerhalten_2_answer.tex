\documentclass[11pt,a4paper]{article}
\usepackage[utf8]{inputenc}
\usepackage[german]{babel}
\usepackage{amsmath}
\usepackage{amsfonts}
\usepackage{subfig}
\usepackage{amssymb}
\usepackage{siunitx,physics}
\usepackage{mathtools}
\usepackage{graphicx}
%\usepackage{Here}
\usepackage[version=4]{mhchem}
\usepackage{url}
\usepackage{setspace}
\usepackage[left=2.5cm,right=2.5cm,top=2.5cm,bottom=2cm]{geometry}
[biblography=totocnumbered]
\usepackage{fancyhdr}
\usepackage{scrextend}
\usepackage{hyperref}
\pagenumbering{gobble}

\makeatletter
\newcommand\bigcdot{\mathpalette\bigcdot@{.5}}
\newcommand\bigcdot@[2]{\mathbin{\vcenter{\hbox{\scalebox{#2}{$\m@th#1\bullet$}}}}}
\makeatother

\makeatletter
%\renewcommand*\bib@heading{%
%  \subsection*{}%
%  \@mkboth{\refname}{\refname}}
%\makeatother
\numberwithin{equation}{section}
\numberwithin{figure}{section}

\renewcommand{\labelitemii}{\labelitemfont$\vartriangleright$}
\begin{document}\\
\begin{addmargin}[25pt]{0pt}
Beim Zugversuch mit angelegter Kraft in z-Richtung wird der Körper in x- und y-Richtung seinen Querschnitt verringern. Dabei wird angenommen, dass die Dehnungen in Querrichtung gilt: $\epsilon_x = \epsilon_y$. Die Poisson-Zahl ist das Verhältnis der Querdehnung zur Längsdehnung:
\begin{equation}\label{eq:Definition_Poisson_Zahl}
\nu = -\frac{\epsilon_x}{\epsilon_z}
\end{equation}
Das negative Vorzeichen kommt daher, dass die Querdehnung kleiner als null ist. Unter der Annahme, dass sich das Volumen des Körpers beim Zugversuch nicht ändert müsste für die Poisson-Zahl $\nu = \frac{1}{2}$ gelten. Allerdings zeigen Experimente, dass diese Annahme nicht berechtigt ist und die Poisson-Zahl für isotrope Materialien ungefähr $\frac{1}{4}$ ist.\\ 
\end{addmargin} 


\end{document}