\documentclass[11pt,a4paper]{article}
\usepackage[utf8]{inputenc}
\usepackage[german]{babel}
\usepackage{amsmath}
\usepackage{amsfonts}
\usepackage{subfig}
\usepackage{amssymb}
\usepackage{siunitx,physics}
\usepackage{mathtools}
\usepackage{graphicx}
%\usepackage{Here}
\usepackage[version=4]{mhchem}
\usepackage{url}
\usepackage{setspace}
\usepackage[left=2.5cm,right=2.5cm,top=2.5cm,bottom=2cm]{geometry}
[biblography=totocnumbered]
\usepackage{fancyhdr}
\usepackage{scrextend}
\usepackage{hyperref}
\pagenumbering{gobble}

\makeatletter
\newcommand\bigcdot{\mathpalette\bigcdot@{.5}}
\newcommand\bigcdot@[2]{\mathbin{\vcenter{\hbox{\scalebox{#2}{$\m@th#1\bullet$}}}}}
\makeatother

\makeatletter
%\renewcommand*\bib@heading{%
%  \subsection*{}%
%  \@mkboth{\refname}{\refname}}
%\makeatother
\numberwithin{equation}{section}
\numberwithin{figure}{section}

\renewcommand{\labelitemii}{\labelitemfont$\vartriangleright$}
\begin{document}\\
\begin{addmargin}[25pt]{0pt}
Poymere können eine Vielzahl von verschiedenen mechanischen Verhaltensweisen aufweisen, somit sind auch so ziemlich alle Spannungs-Dehnungs-Kurven vorstellbar. Die häufigsten auftretenden Kurven sind die des spröden Materials, die des plastischen Materials und die des Elastomers. Das spröde Material kennzeichnet sich durch einen hohen Elastizitätsmodul im elastischen Bereich und danach bricht das Material direkt ohne sich plastisch zu verformen. Das plastische Material hingegen weist nach der elastischen Dehnung ein Fließverhalten auf bevor es zum Bruch kommt. Ein Elastomer lässt sich vollständig elastisch deformieren.\\
\end{addmargin}

\end{document}