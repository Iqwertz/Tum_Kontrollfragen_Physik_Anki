\documentclass[11pt,a4paper]{article}
\usepackage[utf8]{inputenc}
\usepackage[german]{babel}
\usepackage{amsmath}
\usepackage{amsfonts}
\usepackage{subfig}
\usepackage{amssymb}
\usepackage{siunitx,physics}
\usepackage{mathtools}
\usepackage{graphicx}
%\usepackage{Here}
\usepackage[version=4]{mhchem}
\usepackage{url}
\usepackage{setspace}
\usepackage[left=2.5cm,right=2.5cm,top=2.5cm,bottom=2cm]{geometry}
[biblography=totocnumbered]
\usepackage{fancyhdr}
\usepackage{scrextend}
\usepackage{hyperref}
\pagenumbering{gobble}

\makeatletter
\newcommand\bigcdot{\mathpalette\bigcdot@{.5}}
\newcommand\bigcdot@[2]{\mathbin{\vcenter{\hbox{\scalebox{#2}{$\m@th#1\bullet$}}}}}
\makeatother

\makeatletter
%\renewcommand*\bib@heading{%
%  \subsection*{}%
%  \@mkboth{\refname}{\refname}}
%\makeatother
\numberwithin{equation}{section}
\numberwithin{figure}{section}

\renewcommand{\labelitemii}{\labelitemfont$\vartriangleright$}
\begin{document}\\
\begin{addmargin}[25pt]{0pt}
Polymere werden durch den Polymerisationsgrad DP und die Molmasse $M$ charakterisiert. Dabei gibt DP an wie lang die Polymerketten sind, also aus wie vielen Monomeren das Polymer besteht. Der Polymerisationsgrad liegt typischerweise zwischen 100 und 10000. Die Molmasse ist die molare Masse des Polymers, welche bestimmt wird aus der Masse des Monomers $M_{\si{mon}}$ und dem Polymerisationsgrad DP: $M = \si{DP} \cdot M_{\si{mon}}$. Die Eigenschaften des Polymer hängen dabei sehr stark von $M$ ab.  \\
\end{addmargin}

\end{document}