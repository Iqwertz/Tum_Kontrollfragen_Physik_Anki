\documentclass[11pt,a4paper]{article}
\usepackage[utf8]{inputenc}
\usepackage[german]{babel}
\usepackage{amsmath}
\usepackage{amsfonts}
\usepackage{subfig}
\usepackage{amssymb}
\usepackage{siunitx,physics}
\usepackage{mathtools}
\usepackage{graphicx}
%\usepackage{Here}
\usepackage[version=4]{mhchem}
\usepackage{url}
\usepackage{setspace}
\usepackage[left=2.5cm,right=2.5cm,top=2.5cm,bottom=2cm]{geometry}
[biblography=totocnumbered]
\usepackage{fancyhdr}
\usepackage{scrextend}
\usepackage{hyperref}
\pagenumbering{gobble}

\makeatletter
\newcommand\bigcdot{\mathpalette\bigcdot@{.5}}
\newcommand\bigcdot@[2]{\mathbin{\vcenter{\hbox{\scalebox{#2}{$\m@th#1\bullet$}}}}}
\makeatother

\makeatletter
%\renewcommand*\bib@heading{%
%  \subsection*{}%
%  \@mkboth{\refname}{\refname}}
%\makeatother
\numberwithin{equation}{section}
\numberwithin{figure}{section}

\renewcommand{\labelitemii}{\labelitemfont$\vartriangleright$}
\begin{document}\\
\begin{addmargin}[25pt]{0pt}
Ein Tripelpunkt ist der Punkt im Phasendiagramm eines einkomponentigen Systems an dem alle 3 Phasen, fest, flüssig und gasförmig gleichzeitig auftreten und miteinander im Gleichgewicht stehen. Der Tripelpunkt ist exakt beschrieben durch ein Wertepaar aus einer Temperatur und einem Druck. Am Tripelpunkt hat das System also keinen Freiheitsgrad, da alle thermodynamischen Variablen exakt bestimmt sind. Stehen nur 2 Phasen miteinander im Gleichgewicht so hat das System einen Freiheitsgrad, es gibt also für jede beliebige Temperatur genau einen Druck an dem die beiden Phasen im Gleichgewicht stehen können.\\
\end{addmargin}


\end{document}