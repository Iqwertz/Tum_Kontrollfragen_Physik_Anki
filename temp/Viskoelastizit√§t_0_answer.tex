\documentclass[11pt,a4paper]{article}
\usepackage[utf8]{inputenc}
\usepackage[german]{babel}
\usepackage{amsmath}
\usepackage{amsfonts}
\usepackage{subfig}
\usepackage{amssymb}
\usepackage{siunitx,physics}
\usepackage{mathtools}
\usepackage{graphicx}
%\usepackage{Here}
\usepackage[version=4]{mhchem}
\usepackage{url}
\usepackage{setspace}
\usepackage[left=2.5cm,right=2.5cm,top=2.5cm,bottom=2cm]{geometry}
[biblography=totocnumbered]
\usepackage{fancyhdr}
\usepackage{scrextend}
\usepackage{hyperref}
\pagenumbering{gobble}

\makeatletter
\newcommand\bigcdot{\mathpalette\bigcdot@{.5}}
\newcommand\bigcdot@[2]{\mathbin{\vcenter{\hbox{\scalebox{#2}{$\m@th#1\bullet$}}}}}
\makeatother

\makeatletter
%\renewcommand*\bib@heading{%
%  \subsection*{}%
%  \@mkboth{\refname}{\refname}}
%\makeatother
\numberwithin{equation}{section}
\numberwithin{figure}{section}

\renewcommand{\labelitemii}{\labelitemfont$\vartriangleright$}
\begin{document}\\
\begin{addmargin}[25pt]{0pt}
Viskoelastisches Verhalten bedeutet, dass die mechanischen Eigenschaften eines Materials stark temperaturabhängig sind. Viskoelastische Materialien zeigen bei niedrigen Temperaturen ein glasförmiges Verhalten, bei mittleren Temperaturen ein gummiartiges Verhalten und bei hohen Temperaturen verhalten sie sich wie eine viskose Flüssigkeit. Das besondere an einer viskosen Flüssigkeit ist die zeitabhängige Spannung und Dehnung mit der Viskosität $\eta$ gilt für eine viskose Flüssigkeit:
\begin{equation}\label{eq:viskose_Spannung}
    \sigma = \eta \frac{\si{d}\epsilon}{\si{d}t}
\end{equation}
\end{addmargin} 

\end{document}