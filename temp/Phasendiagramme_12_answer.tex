\documentclass[11pt,a4paper]{article}
\usepackage[utf8]{inputenc}
\usepackage[german]{babel}
\usepackage{amsmath}
\usepackage{amsfonts}
\usepackage{subfig}
\usepackage{amssymb}
\usepackage{siunitx,physics}
\usepackage{mathtools}
\usepackage{graphicx}
%\usepackage{Here}
\usepackage[version=4]{mhchem}
\usepackage{url}
\usepackage{setspace}
\usepackage[left=2.5cm,right=2.5cm,top=2.5cm,bottom=2cm]{geometry}
[biblography=totocnumbered]
\usepackage{fancyhdr}
\usepackage{scrextend}
\usepackage{hyperref}
\pagenumbering{gobble}

\makeatletter
\newcommand\bigcdot{\mathpalette\bigcdot@{.5}}
\newcommand\bigcdot@[2]{\mathbin{\vcenter{\hbox{\scalebox{#2}{$\m@th#1\bullet$}}}}}
\makeatother

\makeatletter
%\renewcommand*\bib@heading{%
%  \subsection*{}%
%  \@mkboth{\refname}{\refname}}
%\makeatother
\numberwithin{equation}{section}
\numberwithin{figure}{section}

\renewcommand{\labelitemii}{\labelitemfont$\vartriangleright$}
\begin{document}\\
\begin{addmargin}[25pt]{0pt}
Bei Raumtemperatur liegt Eisen als Ferrit oder $\alpha$-Eisen vor, dieses hat eine bcc-Struktur und Kohlenstoff ist darin nur schlecht löslich da in der bcc-Struktur nur wenig Zwischengitterplätze vorhanden sind auf denen sich Kohlenstoff einlagern könnte. Ferrit ist relativ weich und bis $768^\circ $C ferromagnetisch. Ab einer Temperatur von $912^\circ$C wandelt sich Ferrit allotrop in Austenit oder $\gamma$-Eisen um, dieses besitzt eine fcc-Struktur wodurch sich Kohlenstoff sehr gut in ihm lösen lässt. Austenit ist nicht ferromagnetisch. Im Temperaturbereich $1394^\circ$C bis $1538^\circ$C hat Eisen wieder eine bcc-Struktur und ähnliche Eigenschaften zu Ferrit, daher nennt man diese Form von Eisen auch $\delta$-Ferrit. Bei noch höheren Temperaturen schmilzt reines Eisen.\\
\end{addmargin}

\end{document}