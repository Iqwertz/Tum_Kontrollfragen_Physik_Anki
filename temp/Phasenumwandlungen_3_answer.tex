\documentclass[11pt,a4paper]{article}
\usepackage[utf8]{inputenc}
\usepackage[german]{babel}
\usepackage{amsmath}
\usepackage{amsfonts}
\usepackage{subfig}
\usepackage{amssymb}
\usepackage{siunitx,physics}
\usepackage{mathtools}
\usepackage{graphicx}
%\usepackage{Here}
\usepackage[version=4]{mhchem}
\usepackage{url}
\usepackage{setspace}
\usepackage[left=2.5cm,right=2.5cm,top=2.5cm,bottom=2cm]{geometry}
[biblography=totocnumbered]
\usepackage{fancyhdr}
\usepackage{scrextend}
\usepackage{hyperref}
\pagenumbering{gobble}

\makeatletter
\newcommand\bigcdot{\mathpalette\bigcdot@{.5}}
\newcommand\bigcdot@[2]{\mathbin{\vcenter{\hbox{\scalebox{#2}{$\m@th#1\bullet$}}}}}
\makeatother

\makeatletter
%\renewcommand*\bib@heading{%
%  \subsection*{}%
%  \@mkboth{\refname}{\refname}}
%\makeatother
\numberwithin{equation}{section}
\numberwithin{figure}{section}

\renewcommand{\labelitemii}{\labelitemfont$\vartriangleright$}
\begin{document}\\
\begin{addmargin}[25pt]{0pt}
Bei der Phasenumwandlung von flüssig in fest und der Annahme von homogener Keimbildung hat die freie Enthalpie 2 Anteile, einmal einen Oberflächenanteil und einen Volumenanteil. Der Volumenanteil berüclsichtigt die Differenz der freien Enthalpie in fester und flüssiger Phase pro Volumen, diese Größe ist $\Delta G_V$, multipliziert man diesen Wert mit dem Volumen $\frac{4}{3}\pi r^3$ des Keims erhält man die gesamte Enthalpieänderung aufgrund des Volumens. Der zweite Anteil betrachtet die freie Grenzflächenenthalpie pro Oberfläche des Keims, dieser Wert wird $\gamma$ genannt analog zum Volumen muss hier $\gamma$ einfach mit der Oberfläche $4\pi r^2$ multipliziert werden um die Enthalpieänderung zu bestimmen. Die gesamte Enthalpieänderung durch einen kugelförmigen Keim von Radius $r$ ist:
\begin{equation}\label{eq:enthalpie_homogene_Keimbildung}
    \Delta G = \frac{4}{3}\pi r^3 \Delta G_V + 4\pi r^2 \gamma
\end{equation}
Für die Umwandlung flüssig $\rightarrow$ fest ist allgemein: $\Delta G_V <0$ und $\gamma >0$\\
\end{addmargin}

\end{document}