\documentclass[11pt,a4paper]{article}
\usepackage[utf8]{inputenc}
\usepackage[german]{babel}
\usepackage{amsmath}
\usepackage{amsfonts}
\usepackage{subfig}
\usepackage{amssymb}
\usepackage{siunitx,physics}
\usepackage{mathtools}
\usepackage{graphicx}
%\usepackage{Here}
\usepackage[version=4]{mhchem}
\usepackage{url}
\usepackage{setspace}
\usepackage[left=2.5cm,right=2.5cm,top=2.5cm,bottom=2cm]{geometry}
[biblography=totocnumbered]
\usepackage{fancyhdr}
\usepackage{scrextend}
\usepackage{hyperref}
\pagenumbering{gobble}

\makeatletter
\newcommand\bigcdot{\mathpalette\bigcdot@{.5}}
\newcommand\bigcdot@[2]{\mathbin{\vcenter{\hbox{\scalebox{#2}{$\m@th#1\bullet$}}}}}
\makeatother

\makeatletter
%\renewcommand*\bib@heading{%
%  \subsection*{}%
%  \@mkboth{\refname}{\refname}}
%\makeatother
\numberwithin{equation}{section}
\numberwithin{figure}{section}

\renewcommand{\labelitemii}{\labelitemfont$\vartriangleright$}
\begin{document}
\indent \begin{itemize}
    \item Metalle
    \begin{itemize}
        \item hohe Schlagzähigkeit, Festigkeit, Verformbarkeit
        \item thermisch und elektrisch leitfähig
        \item reaktiv, leicht korrodierend
    \end{itemize}
    \item Polymere
    \begin{itemize}
        \item  niedrige Massendichte, leicht verformbar
        \item geringe Härte, niedriger Elastizitätsmodul
        \item stark druckabhängige Eigenschaften
    \end{itemize}
    \item Elastomere
    \begin{itemize}
        \item hohe Schlagzähigkeit
        \item nehmen ihre Form nach Verformung wieder an
        \item geringe Härte, niedriger Elastizitätsmodul
    \end{itemize}
    \item Gläser
    \begin{itemize}
        \item hart, aber spröde
        \item elektrische Isolatoren
        \item transparent
        \item nicht korrodierend
    \end{itemize}
    \item Kramiken
    \begin{itemize}
        \item hohe Festigkeit, Härte, Abriebfestigkeit 
        \item spröde
        \item nicht korrodierend
    \end{itemize}
\end{itemize}
\newpage
\end{document}