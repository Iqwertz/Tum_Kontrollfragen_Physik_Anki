\documentclass[11pt,a4paper]{article}
\usepackage[utf8]{inputenc}
\usepackage[german]{babel}
\usepackage{amsmath}
\usepackage{amsfonts}
\usepackage{subfig}
\usepackage{amssymb}
\usepackage{siunitx,physics}
\usepackage{mathtools}
\usepackage{graphicx}
%\usepackage{Here}
\usepackage[version=4]{mhchem}
\usepackage{url}
\usepackage{setspace}
\usepackage[left=2.5cm,right=2.5cm,top=2.5cm,bottom=2cm]{geometry}
[biblography=totocnumbered]
\usepackage{fancyhdr}
\usepackage{scrextend}
\usepackage{hyperref}
\pagenumbering{gobble}

\makeatletter
\newcommand\bigcdot{\mathpalette\bigcdot@{.5}}
\newcommand\bigcdot@[2]{\mathbin{\vcenter{\hbox{\scalebox{#2}{$\m@th#1\bullet$}}}}}
\makeatother

\makeatletter
%\renewcommand*\bib@heading{%
%  \subsection*{}%
%  \@mkboth{\refname}{\refname}}
%\makeatother
\numberwithin{equation}{section}
\numberwithin{figure}{section}

\renewcommand{\labelitemii}{\labelitemfont$\vartriangleright$}
\begin{document}\\
\begin{addmargin}[25pt]{0pt}
Aus Abbidung \ref{fig:zweiphasengebiet_konode} kann man nicht nur die Zusammensetzung der Phasen bestimmen sondern auch wie viel der jeweiligen Phase im System vorhanden ist, dafür bestimmt man die Längen der Konoden und nennt diese $R$ und $S$ diese sind auch in Abbildung \ref{fig:zweiphasengebiet_konode} illustriert. Die Massenanteile an fester/flüssiger Phase sind dann:
\begin{align}
    Ma_\alpha &= \frac{R}{R+S}\\
    Ma_L &= \frac{S}{R+S}
\end{align}
Aus dieser und der letzten Frage ergibt sich noch eine interessante Erkenntnis, die Zusammensetzung der Phasen ist nur temperaturabhängig, also wenn man den Punkt B bei gleicher Temperatur nach links oder rechts verschiebt wird sich die Menge an Nickel in fester bzw. flüssiger Phase nicht ändern. Da sich bei einer Verschiebung nach rechts oder links $R$ und $S$ deutlich ändern werden sich die Massenanteile von flüssiger und fester Phase erheblich ändern. Das bedeutet, wenn man bei gleicher Temperatur 2 Kupfer-Nickel-Legierungen (mit unterschiedlichem Nickelgehalt) im Zweiphasengebiet beobachtet so haben die Schmelzen und die Kristalle jeweils genauso viel Nickel allerdings wird bei der Legierung mit dem insgesamt höheren Nickelanteil deutlich mehr feste Phase vorliegen.\\  
\end{addmargin}

\end{document}