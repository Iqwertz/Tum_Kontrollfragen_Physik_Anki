\documentclass[11pt,a4paper]{article}
\usepackage[utf8]{inputenc}
\usepackage[german]{babel}
\usepackage{amsmath}
\usepackage{amsfonts}
\usepackage{subfig}
\usepackage{amssymb}
\usepackage{siunitx,physics}
\usepackage{mathtools}
\usepackage{graphicx}
%\usepackage{Here}
\usepackage[version=4]{mhchem}
\usepackage{url}
\usepackage{setspace}
\usepackage[left=2.5cm,right=2.5cm,top=2.5cm,bottom=2cm]{geometry}
[biblography=totocnumbered]
\usepackage{fancyhdr}
\usepackage{scrextend}
\usepackage{hyperref}
\pagenumbering{gobble}

\makeatletter
\newcommand\bigcdot{\mathpalette\bigcdot@{.5}}
\newcommand\bigcdot@[2]{\mathbin{\vcenter{\hbox{\scalebox{#2}{$\m@th#1\bullet$}}}}}
\makeatother

\makeatletter
%\renewcommand*\bib@heading{%
%  \subsection*{}%
%  \@mkboth{\refname}{\refname}}
%\makeatother
\numberwithin{equation}{section}
\numberwithin{figure}{section}

\renewcommand{\labelitemii}{\labelitemfont$\vartriangleright$}
\begin{document}\\
\begin{addmargin}[25pt]{0pt}
Kohlenstoff kann entweder als Fullerene \ce{C60} vorliegen oder als Kohlenstoff-Nanoröhrchen. Fullerene sind hohle Kugeln, welche aus 60 C-Atomen bestehen die in 5- und 6-Ringen angeordnet sind. Im Festkörper bilden mehrere Fullerene eine bcc-Struktur und wirken als Isolator. Kohlenstoff-Nanoröhrchen sind Graphitschichten, welche zu einer zylindrigen Form aufgerollt wurden. Diese Zylinder werden mit \ce{C60}-Halbkugeln abgedeckt. Kohlenstoff-Nanoröhrchen haben einen Durchmesser von unter 100 nm  und einer Länge die die größer ist als 1000 mal der Durchmesser.\\
\end{addmargin}

\end{document}