\documentclass[11pt,a4paper]{article}
\usepackage[utf8]{inputenc}
\usepackage[german]{babel}
\usepackage{amsmath}
\usepackage{amsfonts}
\usepackage{subfig}
\usepackage{amssymb}
\usepackage{siunitx,physics}
\usepackage{mathtools}
\usepackage{graphicx}
%\usepackage{Here}
\usepackage[version=4]{mhchem}
\usepackage{url}
\usepackage{setspace}
\usepackage[left=2.5cm,right=2.5cm,top=2.5cm,bottom=2cm]{geometry}
[biblography=totocnumbered]
\usepackage{fancyhdr}
\usepackage{scrextend}
\usepackage{hyperref}
\pagenumbering{gobble}

\makeatletter
\newcommand\bigcdot{\mathpalette\bigcdot@{.5}}
\newcommand\bigcdot@[2]{\mathbin{\vcenter{\hbox{\scalebox{#2}{$\m@th#1\bullet$}}}}}
\makeatother

\makeatletter
%\renewcommand*\bib@heading{%
%  \subsection*{}%
%  \@mkboth{\refname}{\refname}}
%\makeatother
\numberwithin{equation}{section}
\numberwithin{figure}{section}

\renewcommand{\labelitemii}{\labelitemfont$\vartriangleright$}
\begin{document}\\
\begin{addmargin}[25pt]{0pt}
Dieser Zusammenhang ist über die Gibbs'sche Phasenregel definiert:
\begin{equation}\label{eq:Gibbs_Phasenregel}
    P + F = K + N
\end{equation}
Dabei ist $P$ die Anzahl der vorliegenden Phasen, $F$ die Anzahl der Freiheitsgrade des Systems, $K$ die Anzahl der Komponenten im System und $N$ die Anzahl der Zustandvariablen. In einem binären System wie einer Kupfer-Silber Legierung ist $K = 2$. Da in unserem Fall der Druck immer als Standarddruck vorgegeben wird, ist die Temperatur die einzige Zustandsvariable, also $N = 1$. Damit ergibt sich für ein Einphasengebiet: $F = 2$ also sind Temperatur und Zusammensetzung unabhängig voneinander. Im Zweiphasengebiet ist $F = 1$ also ist für eine feste Temperatur die Zusammensetzung der festen und der flüssigen Phase vorgegeben, das habe wir bereits in Frage 6 dieses Kapitels gesehen. Im Dreiphasengebiet müssen sowohl Temperatur als auch Zusammensetzung einen bestimmten Wert haben, demnach st $F = 0$.\\ 
\end{addmargin}

\end{document}