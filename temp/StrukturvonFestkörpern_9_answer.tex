\documentclass[11pt,a4paper]{article}
\usepackage[utf8]{inputenc}
\usepackage[german]{babel}
\usepackage{amsmath}
\usepackage{amsfonts}
\usepackage{subfig}
\usepackage{amssymb}
\usepackage{siunitx,physics}
\usepackage{mathtools}
\usepackage{graphicx}
%\usepackage{Here}
\usepackage[version=4]{mhchem}
\usepackage{url}
\usepackage{setspace}
\usepackage[left=2.5cm,right=2.5cm,top=2.5cm,bottom=2cm]{geometry}
[biblography=totocnumbered]
\usepackage{fancyhdr}
\usepackage{scrextend}
\usepackage{hyperref}
\pagenumbering{gobble}

\makeatletter
\newcommand\bigcdot{\mathpalette\bigcdot@{.5}}
\newcommand\bigcdot@[2]{\mathbin{\vcenter{\hbox{\scalebox{#2}{$\m@th#1\bullet$}}}}}
\makeatother

\makeatletter
%\renewcommand*\bib@heading{%
%  \subsection*{}%
%  \@mkboth{\refname}{\refname}}
%\makeatother
\numberwithin{equation}{section}
\numberwithin{figure}{section}

\renewcommand{\labelitemii}{\labelitemfont$\vartriangleright$}
\begin{document}\\
\begin{addmargin}[25pt]{0pt}    
Silikate sind 4-fach negativ geladene Ionen aus einem Silizium- und vier Sauerstoffatomen. Durch die starke negative Ladung der 4 Sauerstoffatome kann diese Molekül sehr starke ionische Bindungen mit metallischen Kationen eingehen. Die beiden wichtigsten Vertreter sind Schichtsilikate und mineralische Gläser. Bei den Schichtsilikaten bilden die Silikate zweidimensionale Schichten aus, welche untereinander durch sekundäre Bindungen von den Silikaten in einem festen Abstand bleiben, dadurch kann zum Beispiel Wasser zwischen diese Schichten einlagern, damit kann man die plastischen Eigenschaften des Tons erklären. Mineralische Gläser sind dreidimensionale Silikatverbindungen, wenn in dieser 3-Struktur zusätzlich Natrium und Calcium eingelagert wird erhält man Fensterglas.   \\
\end{addmargin}









\end{document}