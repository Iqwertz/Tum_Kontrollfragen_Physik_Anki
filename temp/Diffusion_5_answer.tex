\documentclass[11pt,a4paper]{article}
\usepackage[utf8]{inputenc}
\usepackage[german]{babel}
\usepackage{amsmath}
\usepackage{amsfonts}
\usepackage{subfig}
\usepackage{amssymb}
\usepackage{siunitx,physics}
\usepackage{mathtools}
\usepackage{graphicx}
%\usepackage{Here}
\usepackage[version=4]{mhchem}
\usepackage{url}
\usepackage{setspace}
\usepackage[left=2.5cm,right=2.5cm,top=2.5cm,bottom=2cm]{geometry}
[biblography=totocnumbered]
\usepackage{fancyhdr}
\usepackage{scrextend}
\usepackage{hyperref}
\pagenumbering{gobble}

\makeatletter
\newcommand\bigcdot{\mathpalette\bigcdot@{.5}}
\newcommand\bigcdot@[2]{\mathbin{\vcenter{\hbox{\scalebox{#2}{$\m@th#1\bullet$}}}}}
\makeatother

\makeatletter
%\renewcommand*\bib@heading{%
%  \subsection*{}%
%  \@mkboth{\refname}{\refname}}
%\makeatother
\numberwithin{equation}{section}
\numberwithin{figure}{section}

\renewcommand{\labelitemii}{\labelitemfont$\vartriangleright$}
\begin{document}\\
\begin{addmargin}[25pt]{0pt}
An einer Grenzfläche (wie in Abbildung \ref{fig:grenzflächendiffusion} für Kupfer und Nickel zu sehen) sind die Konzentrationen zum Zeitpunkt $t = 0$ scharfe Stufenfunktionen mit der Unstetigkeitsstelle am Ort der Grenzfläche. Mit der Zeit werden allerdings immer mehr Atome von beiden Substanzen über die Grenzfläche diffundieren und so wird sich die Konzentration auch verändern. Die neue Funktion des Konzentrationsprofils ist eine von Ort und Zeit abhängige Gauß'sche Fehlerfunktion. \\

\begin{figure}[h]
    \centering
    \includegraphics[width = 0.9\textwidth]{images/Materialwissenschaften/Grenzflächendiffsuion.jpeg}
    \caption{Diffusion an einer Cu-Ni-Grenzfläche zum Zeitpunkt 0 und zu einem späteren Zeitpunkt }
    \label{fig:grenzflächendiffusion}
\end{figure}
\end{addmargin} 

\end{document}