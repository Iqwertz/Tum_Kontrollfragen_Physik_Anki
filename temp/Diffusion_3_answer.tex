\documentclass[11pt,a4paper]{article}
\usepackage[utf8]{inputenc}
\usepackage[german]{babel}
\usepackage{amsmath}
\usepackage{amsfonts}
\usepackage{subfig}
\usepackage{amssymb}
\usepackage{siunitx,physics}
\usepackage{mathtools}
\usepackage{graphicx}
%\usepackage{Here}
\usepackage[version=4]{mhchem}
\usepackage{url}
\usepackage{setspace}
\usepackage[left=2.5cm,right=2.5cm,top=2.5cm,bottom=2cm]{geometry}
[biblography=totocnumbered]
\usepackage{fancyhdr}
\usepackage{scrextend}
\usepackage{hyperref}
\pagenumbering{gobble}

\makeatletter
\newcommand\bigcdot{\mathpalette\bigcdot@{.5}}
\newcommand\bigcdot@[2]{\mathbin{\vcenter{\hbox{\scalebox{#2}{$\m@th#1\bullet$}}}}}
\makeatother

\makeatletter
%\renewcommand*\bib@heading{%
%  \subsection*{}%
%  \@mkboth{\refname}{\refname}}
%\makeatother
\numberwithin{equation}{section}
\numberwithin{figure}{section}

\renewcommand{\labelitemii}{\labelitemfont$\vartriangleright$}
\begin{document}\\
\begin{addmargin}[25pt]{0pt}
Die Diffusionsstromdichte ist proportional zum Konzentrationsgradienten. 
\begin{equation}\label{eq:Fick1}
    \mathbf{J}(x,y,z) = -D\cdot \nabla c(x,y,z) 
\end{equation}
Die Proportionalitätskonstante ist der Diffusionskoeffizient $D$ mit der Einheit $\left[ \frac{\si{m}^2}{\si{s}}\right]$. Der Diffusionskoeffizient ist eine Stoffeigenschaft die von den beiden beteiligten Stoffen abhängt. Den Zusammenhang in Gleichung \ref{eq:Fick1} nennt man \textit{1. Fick'sches Gesetz}. Das Minuszeichen in dieser Gleichung besagt, dass der Diffusionsstrom von Orten hoher Konzentration zu Orten niedriger Konzentration geht.\\
\end{addmargin} 

\end{document}