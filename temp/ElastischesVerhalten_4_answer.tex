\documentclass[11pt,a4paper]{article}
\usepackage[utf8]{inputenc}
\usepackage[german]{babel}
\usepackage{amsmath}
\usepackage{amsfonts}
\usepackage{subfig}
\usepackage{amssymb}
\usepackage{siunitx,physics}
\usepackage{mathtools}
\usepackage{graphicx}
%\usepackage{Here}
\usepackage[version=4]{mhchem}
\usepackage{url}
\usepackage{setspace}
\usepackage[left=2.5cm,right=2.5cm,top=2.5cm,bottom=2cm]{geometry}
[biblography=totocnumbered]
\usepackage{fancyhdr}
\usepackage{scrextend}
\usepackage{hyperref}
\pagenumbering{gobble}

\makeatletter
\newcommand\bigcdot{\mathpalette\bigcdot@{.5}}
\newcommand\bigcdot@[2]{\mathbin{\vcenter{\hbox{\scalebox{#2}{$\m@th#1\bullet$}}}}}
\makeatother

\makeatletter
%\renewcommand*\bib@heading{%
%  \subsection*{}%
%  \@mkboth{\refname}{\refname}}
%\makeatother
\numberwithin{equation}{section}
\numberwithin{figure}{section}

\renewcommand{\labelitemii}{\labelitemfont$\vartriangleright$}
\begin{document}\\
\begin{addmargin}[25pt]{0pt}
Streckt man einen Körper um eine Länge $\si{d}l$ so muss man Arbeit verrichten um der Rückstellkraft $F_r$ des elastischen Zusammenziehens entgegenzuwirken. Eine zweite Komponente der geleisteten Arbeit tritt auf, da man beim Dehnen das Volumen des Körpers ändert und so Volumenänderungsarbeit $p\si{d}V$ verrichten muss, die gesamte aufzubringende Arbeit $\si{d}W$ ist dann:
\begin{equation}\label{eq:geleistete_Arbeit_Zugversuch}
    \delta W = -F_r \si{d}l + p\si{d}V
\end{equation}
Da man von reversiblem Dehnen ausgeht kann man auch noch die transportierte Wärme $\delta Q = T\si{d}S$ bestimmen und so mit dem ersten Hauptsatz der Thermodynamik die Änderung der inneren Energie $\si{d}U$ des Systems:
\begin{equation}\label{eq:innere_Energie_Zugversuch}
    \si{d}U = \delta Q - \delta W = T\si{d}S + F_r \si{d}l - p\si{d}V
\end{equation}
\end{addmargin} 


\end{document}