\documentclass[11pt,a4paper]{article}
\usepackage[utf8]{inputenc}
\usepackage[german]{babel}
\usepackage{amsmath}
\usepackage{amsfonts}
\usepackage{subfig}
\usepackage{amssymb}
\usepackage{siunitx,physics}
\usepackage{mathtools}
\usepackage{graphicx}
%\usepackage{Here}
\usepackage[version=4]{mhchem}
\usepackage{url}
\usepackage{setspace}
\usepackage[left=2.5cm,right=2.5cm,top=2.5cm,bottom=2cm]{geometry}
[biblography=totocnumbered]
\usepackage{fancyhdr}
\usepackage{scrextend}
\usepackage{hyperref}
\pagenumbering{gobble}

\makeatletter
\newcommand\bigcdot{\mathpalette\bigcdot@{.5}}
\newcommand\bigcdot@[2]{\mathbin{\vcenter{\hbox{\scalebox{#2}{$\m@th#1\bullet$}}}}}
\makeatother

\makeatletter
%\renewcommand*\bib@heading{%
%  \subsection*{}%
%  \@mkboth{\refname}{\refname}}
%\makeatother
\numberwithin{equation}{section}
\numberwithin{figure}{section}

\renewcommand{\labelitemii}{\labelitemfont$\vartriangleright$}
\begin{document}\\
\begin{addmargin}[25pt]{0pt}     
Die metallische Bindung ist stark und nicht-orientiert, dadurch bilden Metalle bevorzugt Kristallgitter aus, welche hohe Packungsdichten aufweisen. Die häufigsten Strukturen sind das hcp- und das fcc-Gitter, diese haben jeweils die Koordinationszahl 12 und die Packungsdichte 0,75. In der hcp-Struktur befinden sich 6 Gitterpunkte in der konventionellen Einheitszelle, wichtige Elemente die in der hcp-Struktur kristallisieren sind Magnesium, $\alpha$-Titan und Zink. In der fcc-Struktur hingegen befinden sich 4 Gitterpunkte in der konventionellen Einheitszelle, wichtige Vertreter dieser Struktur sind Aluminium, $\gamma$-Eisen, Kupfer und Nickel. Außerdem ist die bcc-Struktur erwähnenswert, diese hat zwar nur eine Packungsdichte von 0,68 allerdings kristallisieren dennoch einige Metalle in dieser Struktur wie zum Beispiel $\alpha$-Eisen, Molybdän, $\beta$-Titan oder Chrom. \\
\end{addmargin}

\end{document}