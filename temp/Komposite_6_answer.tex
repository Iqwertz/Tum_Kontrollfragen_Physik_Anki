\documentclass[11pt,a4paper]{article}
\usepackage[utf8]{inputenc}
\usepackage[german]{babel}
\usepackage{amsmath}
\usepackage{amsfonts}
\usepackage{subfig}
\usepackage{amssymb}
\usepackage{siunitx,physics}
\usepackage{mathtools}
\usepackage{graphicx}
%\usepackage{Here}
\usepackage[version=4]{mhchem}
\usepackage{url}
\usepackage{setspace}
\usepackage[left=2.5cm,right=2.5cm,top=2.5cm,bottom=2cm]{geometry}
[biblography=totocnumbered]
\usepackage{fancyhdr}
\usepackage{scrextend}
\usepackage{hyperref}
\pagenumbering{gobble}

\makeatletter
\newcommand\bigcdot{\mathpalette\bigcdot@{.5}}
\newcommand\bigcdot@[2]{\mathbin{\vcenter{\hbox{\scalebox{#2}{$\m@th#1\bullet$}}}}}
\makeatother

\makeatletter
%\renewcommand*\bib@heading{%
%  \subsection*{}%
%  \@mkboth{\refname}{\refname}}
%\makeatother
\numberwithin{equation}{section}
\numberwithin{figure}{section}

\renewcommand{\labelitemii}{\labelitemfont$\vartriangleright$}
\begin{document}\\
\begin{addmargin}[25pt]{0pt}
Kohlenstoff-Nanoröhrchen haben untereinander eine starke Anziehung aufgrund der van-der-Waals-Wechselwirkung, dadurch bilden sich bevorzugt Cluster welche in Nanokompositen nicht gewünscht sind. Obwohl CNTs zur Matrix eine große Kontaktfläche haben, wodurch sich eine gute Leitfähigkeit ergibt, so sind sie dennoch schlecht an sie angebunden, dadurch werden sie bei Belastung eventuell mechanisch herausgezogen. Diese Schwierigkeit versucht man zu lösen indem man die Grenzfläche mit chemischen Reaktionen an die Anwendung anpasst.  \\
\end{addmargin}

\end{document}