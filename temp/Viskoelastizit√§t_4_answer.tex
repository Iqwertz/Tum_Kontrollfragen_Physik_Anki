\documentclass[11pt,a4paper]{article}
\usepackage[utf8]{inputenc}
\usepackage[german]{babel}
\usepackage{amsmath}
\usepackage{amsfonts}
\usepackage{subfig}
\usepackage{amssymb}
\usepackage{siunitx,physics}
\usepackage{mathtools}
\usepackage{graphicx}
%\usepackage{Here}
\usepackage[version=4]{mhchem}
\usepackage{url}
\usepackage{setspace}
\usepackage[left=2.5cm,right=2.5cm,top=2.5cm,bottom=2cm]{geometry}
[biblography=totocnumbered]
\usepackage{fancyhdr}
\usepackage{scrextend}
\usepackage{hyperref}
\pagenumbering{gobble}

\makeatletter
\newcommand\bigcdot{\mathpalette\bigcdot@{.5}}
\newcommand\bigcdot@[2]{\mathbin{\vcenter{\hbox{\scalebox{#2}{$\m@th#1\bullet$}}}}}
\makeatother

\makeatletter
%\renewcommand*\bib@heading{%
%  \subsection*{}%
%  \@mkboth{\refname}{\refname}}
%\makeatother
\numberwithin{equation}{section}
\numberwithin{figure}{section}

\renewcommand{\labelitemii}{\labelitemfont$\vartriangleright$}
\begin{document}\\
\begin{addmargin}[25pt]{0pt}
Um eine viskoelastische Flüssigkeit zu beschreiben verwendet man das Maxwell-Modell. In diesem Modell wird eine Feder mit einem Dämpfer in Reihe geschalten, dadurch verhält es auf kurze Zeiten elastisch und auf lange Zeiten viskos. Die Gesamtdehnung in diesem Modell ist die Summe der Einzeldehnungen. Für eine viskoelastischen Festkörper nutzt man das Voigt-Kelvin-Modell bei dem eine Feder und ein Dämpfer parallel geschalten sind. Hierbei ist die Spannung die Summe der Einzelspannungen. Auf kurze Zeit verhält dieses Modell viskos und auf lange Zeit elastisch, genau so wie man das von einem viskoelastischen Festkörper erwartet.\\
\end{addmargin}


\end{document}