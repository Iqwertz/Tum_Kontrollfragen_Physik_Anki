\documentclass[11pt,a4paper]{article}
\usepackage[utf8]{inputenc}
\usepackage[german]{babel}
\usepackage{amsmath}
\usepackage{amsfonts}
\usepackage{subfig}
\usepackage{amssymb}
\usepackage{siunitx,physics}
\usepackage{mathtools}
\usepackage{graphicx}
%\usepackage{Here}
\usepackage[version=4]{mhchem}
\usepackage{url}
\usepackage{setspace}
\usepackage[left=2.5cm,right=2.5cm,top=2.5cm,bottom=2cm]{geometry}
[biblography=totocnumbered]
\usepackage{fancyhdr}
\usepackage{scrextend}
\usepackage{hyperref}
\pagenumbering{gobble}

\makeatletter
\newcommand\bigcdot{\mathpalette\bigcdot@{.5}}
\newcommand\bigcdot@[2]{\mathbin{\vcenter{\hbox{\scalebox{#2}{$\m@th#1\bullet$}}}}}
\makeatother

\makeatletter
%\renewcommand*\bib@heading{%
%  \subsection*{}%
%  \@mkboth{\refname}{\refname}}
%\makeatother
\numberwithin{equation}{section}
\numberwithin{figure}{section}

\renewcommand{\labelitemii}{\labelitemfont$\vartriangleright$}
\begin{document}\\
\begin{addmargin}[25pt]{0pt}
Wenn ein Probenkörper über die Streckgrenze gedehnt wurde, dann treten plastische, irreversible Verformungen auf. Lässt man nun den Körper entspannen so wird er eine Dehnung aufweisen obwohl keine Spannung mehr anliegt. Diese Erholung des Materials geschieht auf einer Geraden parallel zum ursprünglichen elastischen Bereich. Falls man nun eine erneute Spannung an den bereits plastisch verformten Körper anlegt, so wird er wieder entlang dieser elastischen Geraden auf die Spannungs-Dehnungs-Kurve zurückkehren. Dieses Materialverhalten nennt man elastische Erholung.\\
\end{addmargin}

\end{document}