\documentclass[11pt,a4paper]{article}
\usepackage[utf8]{inputenc}
\usepackage[german]{babel}
\usepackage{amsmath}
\usepackage{amsfonts}
\usepackage{subfig}
\usepackage{amssymb}
\usepackage{siunitx,physics}
\usepackage{mathtools}
\usepackage{graphicx}
%\usepackage{Here}
\usepackage[version=4]{mhchem}
\usepackage{url}
\usepackage{setspace}
\usepackage[left=2.5cm,right=2.5cm,top=2.5cm,bottom=2cm]{geometry}
[biblography=totocnumbered]
\usepackage{fancyhdr}
\usepackage{scrextend}
\usepackage{hyperref}
\pagenumbering{gobble}

\makeatletter
\newcommand\bigcdot{\mathpalette\bigcdot@{.5}}
\newcommand\bigcdot@[2]{\mathbin{\vcenter{\hbox{\scalebox{#2}{$\m@th#1\bullet$}}}}}
\makeatother

\makeatletter
%\renewcommand*\bib@heading{%
%  \subsection*{}%
%  \@mkboth{\refname}{\refname}}
%\makeatother
\numberwithin{equation}{section}
\numberwithin{figure}{section}

\renewcommand{\labelitemii}{\labelitemfont$\vartriangleright$}
\begin{document}\\
\begin{addmargin}[25pt]{0pt}
Im kristallinen Bereich von teilkristallinen Polymeren sind die Kettenstücke parallel angeordnet und gefaltet. Dadurch bilden sich 10 bis 20 Nanometer dünne Plättchen mit einer lateralen Ausdehnung von 10 bis 50 Mikrometer, diese Plättchen nennt man Kristallite. Kristallite ordnen sich zu Sphäroliten zusammen, das sind kugelförmige Objekte welche von einem Kristallkern aus wachsen. An diesen Kristallkern lagern sich mehrere Kristallite kettenartig an, in den Zwischenräumen zwischen den Kristallitketten bilden sich amorphe Bereiche. \\
\end{addmargin}

\end{document}