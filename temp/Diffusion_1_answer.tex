\documentclass[11pt,a4paper]{article}
\usepackage[utf8]{inputenc}
\usepackage[german]{babel}
\usepackage{amsmath}
\usepackage{amsfonts}
\usepackage{subfig}
\usepackage{amssymb}
\usepackage{siunitx,physics}
\usepackage{mathtools}
\usepackage{graphicx}
%\usepackage{Here}
\usepackage[version=4]{mhchem}
\usepackage{url}
\usepackage{setspace}
\usepackage[left=2.5cm,right=2.5cm,top=2.5cm,bottom=2cm]{geometry}
[biblography=totocnumbered]
\usepackage{fancyhdr}
\usepackage{scrextend}
\usepackage{hyperref}
\pagenumbering{gobble}

\makeatletter
\newcommand\bigcdot{\mathpalette\bigcdot@{.5}}
\newcommand\bigcdot@[2]{\mathbin{\vcenter{\hbox{\scalebox{#2}{$\m@th#1\bullet$}}}}}
\makeatother

\makeatletter
%\renewcommand*\bib@heading{%
%  \subsection*{}%
%  \@mkboth{\refname}{\refname}}
%\makeatother
\numberwithin{equation}{section}
\numberwithin{figure}{section}

\renewcommand{\labelitemii}{\labelitemfont$\vartriangleright$}
\begin{document}\\
\begin{addmargin}[25pt]{0pt}
Im Wesentlichen werden 2 Arten der Diffusion unterschieden. Die erste Variante ist die Leerstellendiffusion, bei dieser springt ein Atom von seinem Gitterplatz zu einem benachbarten leeren Gitterplatz. Das geschieht zum Beispiel in Metallen bei hohen Temperaturen. Die zweite Möglichkeit ist die interstitielle Diffusion, bei dieser bewegt sich ein Zwischengitteratom von einem Zwischengitterplatz zum Nächsten.  Dieser Mechanismus geschieht bei Metallen mit kleinen Fremdatomen wie Wasserstoff, Kohlenstoff, Sauerstoff oder Stickstoff.\\
\end{addmargin} 

\end{document}