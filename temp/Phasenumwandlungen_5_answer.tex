\documentclass[11pt,a4paper]{article}
\usepackage[utf8]{inputenc}
\usepackage[german]{babel}
\usepackage{amsmath}
\usepackage{amsfonts}
\usepackage{subfig}
\usepackage{amssymb}
\usepackage{siunitx,physics}
\usepackage{mathtools}
\usepackage{graphicx}
%\usepackage{Here}
\usepackage[version=4]{mhchem}
\usepackage{url}
\usepackage{setspace}
\usepackage[left=2.5cm,right=2.5cm,top=2.5cm,bottom=2cm]{geometry}
[biblography=totocnumbered]
\usepackage{fancyhdr}
\usepackage{scrextend}
\usepackage{hyperref}
\pagenumbering{gobble}

\makeatletter
\newcommand\bigcdot{\mathpalette\bigcdot@{.5}}
\newcommand\bigcdot@[2]{\mathbin{\vcenter{\hbox{\scalebox{#2}{$\m@th#1\bullet$}}}}}
\makeatother

\makeatletter
%\renewcommand*\bib@heading{%
%  \subsection*{}%
%  \@mkboth{\refname}{\refname}}
%\makeatother
\numberwithin{equation}{section}
\numberwithin{figure}{section}

\renewcommand{\labelitemii}{\labelitemfont$\vartriangleright$}
\begin{document}\\
\begin{addmargin}[25pt]{0pt}
In den Gleichungen \ref{eq:kritischer_Radius} und \ref{eq:Aktivierungsenergie} ist $\Delta G_V$ temperaturabhängig und $\gamma$ nicht. Für die Temperaturabhängigkeit von $\Delta G_V$ nutzt man folgenden Zusammenhang:
\begin{equation}\label{eq:Umwandlungswärme}
    \Delta G_V = \frac{\Delta H_f \cdot (T_S - T)}{T_S}
\end{equation}
dabei ist $T_S$ die Schmelztemperatur und $\Delta H_f$ die bei Erstarrung freigesetzte Wärmemenge. Eingesetzt in die Gleichungen  \ref{eq:kritischer_Radius} und \ref{eq:Aktivierungsenergie} ergibt das:
\begin{align}
    r^* &= \left( - \frac{2\gamma T_S}{\Delta H_f}\right) \frac{1}{T_S - T}\\
    \Delta G^* &= \left( \frac{16\pi \gamma^3 T_S^2}{3(\Delta H_f)^2}\right) \frac{1}{(T_S - T)^2} 
\end{align}
Die Anzahl der stabilen Keime (für diese gilt: $r>r^*$) ist gegeben mit:
\begin{equation}
    n^* = K_1 \exp(-\frac{\Delta G^*}{k_BT})
\end{equation}
\end{addmargin}

\end{document}