\documentclass[11pt,a4paper]{article}
\usepackage[utf8]{inputenc}
\usepackage[german]{babel}
\usepackage{amsmath}
\usepackage{amsfonts}
\usepackage{subfig}
\usepackage{amssymb}
\usepackage{siunitx,physics}
\usepackage{mathtools}
\usepackage{graphicx}
%\usepackage{Here}
\usepackage[version=4]{mhchem}
\usepackage{url}
\usepackage{setspace}
\usepackage[left=2.5cm,right=2.5cm,top=2.5cm,bottom=2cm]{geometry}
[biblography=totocnumbered]
\usepackage{fancyhdr}
\usepackage{scrextend}
\usepackage{hyperref}
\pagenumbering{gobble}

\makeatletter
\newcommand\bigcdot{\mathpalette\bigcdot@{.5}}
\newcommand\bigcdot@[2]{\mathbin{\vcenter{\hbox{\scalebox{#2}{$\m@th#1\bullet$}}}}}
\makeatother

\makeatletter
%\renewcommand*\bib@heading{%
%  \subsection*{}%
%  \@mkboth{\refname}{\refname}}
%\makeatother
\numberwithin{equation}{section}
\numberwithin{figure}{section}

\renewcommand{\labelitemii}{\labelitemfont$\vartriangleright$}
\begin{document}\\
\begin{addmargin}[25pt]{0pt}
Die Bruchzähigkeit $K_c$ ist definiert als:
\begin{equation}\label{eq:definition_Bruchzähigkeit}
    K_c = Y\sigma_c \sqrt{\pi a}
\end{equation}
dabei ist $Y$ ein dimensionsloser Parameter, welcher die Geometrie des Risses und des Bauteils sowie die Art der Belastung beinhaltet, $\sigma_c$ ist die kritische Spannung für die Ausbreitung eines Risses in einem spröden Material und ist definiert als:
\begin{equation}\label{eq:definition_kritische_Spannung_sprödes_Material}
    \sigma_c = \left( \frac{2E\gamma_s}{\pi a}\right)^\frac{1}{2}
\end{equation}
dabei ist $E$ der Elastizitätsmodul und $\gamma_s$ die spezifische Bruchflächenenergie. Bei Beanspruchung nach Mode I, das ist eine Öffnungs- oder Zugbeanspruchung, ist die Bruchzähigkeit für spröde Materialien niedrig und für duktile Materialien hoch.\\
\end{addmargin}

\end{document}