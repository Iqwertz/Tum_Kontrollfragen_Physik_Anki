\documentclass[11pt,a4paper]{article}
\usepackage[utf8]{inputenc}
\usepackage[german]{babel}
\usepackage{amsmath}
\usepackage{amsfonts}
\usepackage{subfig}
\usepackage{amssymb}
\usepackage{siunitx,physics}
\usepackage{mathtools}
\usepackage{graphicx}
%\usepackage{Here}
\usepackage[version=4]{mhchem}
\usepackage{url}
\usepackage{setspace}
\usepackage[left=2.5cm,right=2.5cm,top=2.5cm,bottom=2cm]{geometry}
[biblography=totocnumbered]
\usepackage{fancyhdr}
\usepackage{scrextend}
\usepackage{hyperref}
\pagenumbering{gobble}

\makeatletter
\newcommand\bigcdot{\mathpalette\bigcdot@{.5}}
\newcommand\bigcdot@[2]{\mathbin{\vcenter{\hbox{\scalebox{#2}{$\m@th#1\bullet$}}}}}
\makeatother

\makeatletter
%\renewcommand*\bib@heading{%
%  \subsection*{}%
%  \@mkboth{\refname}{\refname}}
%\makeatother
\numberwithin{equation}{section}
\numberwithin{figure}{section}

\renewcommand{\labelitemii}{\labelitemfont$\vartriangleright$}
\begin{document}\\
\begin{addmargin}[25pt]{0pt}    
In einem Kristallgitter ist die Packungsdichte immer kleiner als 1, also in jedem Metallgitter ist ein kleines, freies Volumen zwischen den einzelnen Atomen, in diesen Bereich können kleinere Fremdatome eingelagert werden. Diese Poistionen zwischen den eigentlichen Gitterplätzen nennt man Zwischengitterplätze. Man unterscheidet zwischen oktahedralen und tetrahedralen Plätzen, dabei ist oktahedral der Platz in der Mitte von 6 Atomen die in einem Oktaeder angeordnet sind und analog ist tetrahedral der Platz in der Mitte von 4 Atomen an den Ecken eines Tetraeders.\\
\end{addmargin}

\end{document}