\documentclass[11pt,a4paper]{article}
\usepackage[utf8]{inputenc}
\usepackage[german]{babel}
\usepackage{amsmath}
\usepackage{amsfonts}
\usepackage{subfig}
\usepackage{amssymb}
\usepackage{siunitx,physics}
\usepackage{mathtools}
\usepackage{graphicx}
%\usepackage{Here}
\usepackage[version=4]{mhchem}
\usepackage{url}
\usepackage{setspace}
\usepackage[left=2.5cm,right=2.5cm,top=2.5cm,bottom=2cm]{geometry}
[biblography=totocnumbered]
\usepackage{fancyhdr}
\usepackage{scrextend}
\usepackage{hyperref}
\pagenumbering{gobble}

\makeatletter
\newcommand\bigcdot{\mathpalette\bigcdot@{.5}}
\newcommand\bigcdot@[2]{\mathbin{\vcenter{\hbox{\scalebox{#2}{$\m@th#1\bullet$}}}}}
\makeatother

\makeatletter
%\renewcommand*\bib@heading{%
%  \subsection*{}%
%  \@mkboth{\refname}{\refname}}
%\makeatother
\numberwithin{equation}{section}
\numberwithin{figure}{section}

\renewcommand{\labelitemii}{\labelitemfont$\vartriangleright$}
\begin{document}\\
\begin{addmargin}[25pt]{0pt}
Es gibt diffusionsabhängige Phasenübergänge bei denen die Anzahl oder die Zusammensetzung der Phasen unverändert bleibt, dann gibt es diffusionsabhängige Umwandlungen bei denen sich die Phasenzusammensetzung und häufig auch die Anzahl der Phasen ändert und schließlich gibt es noch Umwandlungen ohne Diffusionin eine metastabile Phase. Die Erstarrung eines reinen Metalls oder das Kornwachstum sind Beipsiele für Phasenübergänge bei denen die Anzahl oder die Zusammensetzung der Phasen unverändert bleibt. Die eutektoide Reaktion ist ein Beispiel für eine Änderung der Phasenzusammensetzung.\\
\end{addmargin}

\end{document}