\documentclass[11pt,a4paper]{article}
\usepackage[utf8]{inputenc}
\usepackage[german]{babel}
\usepackage{amsmath}
\usepackage{amsfonts}
\usepackage{subfig}
\usepackage{amssymb}
\usepackage{siunitx,physics}
\usepackage{mathtools}
\usepackage{graphicx}
%\usepackage{Here}
\usepackage[version=4]{mhchem}
\usepackage{url}
\usepackage{setspace}
\usepackage[left=2.5cm,right=2.5cm,top=2.5cm,bottom=2cm]{geometry}
[biblography=totocnumbered]
\usepackage{fancyhdr}
\usepackage{scrextend}
\usepackage{hyperref}
\pagenumbering{gobble}

\makeatletter
\newcommand\bigcdot{\mathpalette\bigcdot@{.5}}
\newcommand\bigcdot@[2]{\mathbin{\vcenter{\hbox{\scalebox{#2}{$\m@th#1\bullet$}}}}}
\makeatother

\makeatletter
%\renewcommand*\bib@heading{%
%  \subsection*{}%
%  \@mkboth{\refname}{\refname}}
%\makeatother
\numberwithin{equation}{section}
\numberwithin{figure}{section}

\renewcommand{\labelitemii}{\labelitemfont$\vartriangleright$}
\begin{document}\\
\begin{addmargin}[25pt]{0pt}     
Polymere sind au langen \glqq zick-zack-förmigen\grqq Molekülen aufgebaut, dessen Hauptkette aus Kohlenstoffbindungen besteht. Diese langen Ketten können mit sich selbst vernetzen wodurch das Polymer verformbar und bei schneller Deformation elastisch wird. Verschiedene Arten der Vernetzung bewirken unterschiedliche Eigenschaften des Polymers, zum Beispiel haben Thermoplaste die Eigenschaft bei hohen Temperaturen verformbar zu sein bei niedrigen jedoch nicht.   \\
\end{addmargin}

\end{document}