\documentclass[11pt,a4paper]{article}
\usepackage[utf8]{inputenc}
\usepackage[german]{babel}
\usepackage{amsmath}
\usepackage{amsfonts}
\usepackage{subfig}
\usepackage{amssymb}
\usepackage{siunitx,physics}
\usepackage{mathtools}
\usepackage{graphicx}
%\usepackage{Here}
\usepackage[version=4]{mhchem}
\usepackage{url}
\usepackage{setspace}
\usepackage[left=2.5cm,right=2.5cm,top=2.5cm,bottom=2cm]{geometry}
[biblography=totocnumbered]
\usepackage{fancyhdr}
\usepackage{scrextend}
\usepackage{hyperref}
\pagenumbering{gobble}

\makeatletter
\newcommand\bigcdot{\mathpalette\bigcdot@{.5}}
\newcommand\bigcdot@[2]{\mathbin{\vcenter{\hbox{\scalebox{#2}{$\m@th#1\bullet$}}}}}
\makeatother

\makeatletter
%\renewcommand*\bib@heading{%
%  \subsection*{}%
%  \@mkboth{\refname}{\refname}}
%\makeatother
\numberwithin{equation}{section}
\numberwithin{figure}{section}

\renewcommand{\labelitemii}{\labelitemfont$\vartriangleright$}
\begin{document}\\
\begin{addmargin}[25pt]{0pt}
Das zweite Fick'sche Gesetz lautet in einer Dimension:
\begin{equation}\label{eq:Fick2}
    \frac{\partial c}{\partial t} + \frac{\partial J}{\partial x} = 0
\end{equation}
In 3 Dimensionen wird die örtlichen Ableitung in Gleichung \ref{eq:Fick2} mit dem Gradienten ersetzt. Das zweite Fick'sche Gesetz besagt, dass die Differenz der Teilchenströme die in ein Volumen hinen- bzw. hinausfließen der Änderung der Konzentration im Volumen entspricht.\\
Man kann \ref{eq:Fick2} und Gleichung \ref{eq:Fick1} kombinieren un derhält eine Alternative Formulierung für das zweite Fick'sche Gesetz:
\begin{equation}\label{eq:Fick2_Alternative}
    \frac{\partial c}{\partial t} = D \Delta c
\end{equation}
Dabei ist $\Delta$ der Laplace-Operator der die zweiten örtlichen Ableitungen erhält. Das zweite Fick'sche Gesetz ist also eine partielle Differentialgleichung für die Stoffmengenkonzentration.\\
\end{addmargin} 

\end{document}