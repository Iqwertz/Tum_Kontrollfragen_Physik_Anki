\documentclass[11pt,a4paper]{article}
\usepackage[utf8]{inputenc}
\usepackage[german]{babel}
\usepackage{amsmath}
\usepackage{amsfonts}
\usepackage{subfig}
\usepackage{amssymb}
\usepackage{siunitx,physics}
\usepackage{mathtools}
\usepackage{graphicx}
%\usepackage{Here}
\usepackage[version=4]{mhchem}
\usepackage{url}
\usepackage{setspace}
\usepackage[left=2.5cm,right=2.5cm,top=2.5cm,bottom=2cm]{geometry}
[biblography=totocnumbered]
\usepackage{fancyhdr}
\usepackage{scrextend}
\usepackage{hyperref}
\pagenumbering{gobble}

\makeatletter
\newcommand\bigcdot{\mathpalette\bigcdot@{.5}}
\newcommand\bigcdot@[2]{\mathbin{\vcenter{\hbox{\scalebox{#2}{$\m@th#1\bullet$}}}}}
\makeatother

\makeatletter
%\renewcommand*\bib@heading{%
%  \subsection*{}%
%  \@mkboth{\refname}{\refname}}
%\makeatother
\numberwithin{equation}{section}
\numberwithin{figure}{section}

\renewcommand{\labelitemii}{\labelitemfont$\vartriangleright$}
\begin{document}\\
\begin{addmargin}[25pt]{0pt}
Ein Phasendiagramm eines binären eutektischen Systems ist in Abbildung \ref{fig:Phasendiagramm_eutektisch} gezeigt. Eutektische Systeme zeichnen sich dadurch aus, dass sie 3 Phasen besitzen, zwei feste und eine flüssige. Die festen Phasen sind hier $\alpha$ und $\beta$ und die flüssige Phase wird als $L$ bezeichnet. Da es 3 Phasen gibt, existieren auch 3 Zweiphasengebiete. Ein weiteres Merkmal von eutektischen Systemen ist, dass die Schmelztemperatur der reinen Materialien höher ist als der Legierungen, die geringste Schmelztemperatur tritt am eutektischen Punkt $E$ auf. An diesem Punkt wird bei konstanter Temperatur die gesamte Schmelze in die beiden festen Phasen umgewandelt. Es gibt kein Dreiphasengebiet.\\   
\begin{figure}[h]
    \centering
    \includegraphics[width = 0.8\textwidth]{images/Materialwissenschaften/eutektisch_Phasendiagramm.jpeg}
    \caption{Phasendiagramm eines binären eutektischen Systems.}
    \label{fig:Phasendiagramm_eutektisch}
\end{figure}
\end{addmargin}

\end{document}