\documentclass[11pt,a4paper]{article}
\usepackage[utf8]{inputenc}
\usepackage[german]{babel}
\usepackage{amsmath}
\usepackage{amsfonts}
\usepackage{subfig}
\usepackage{amssymb}
\usepackage{siunitx,physics}
\usepackage{mathtools}
\usepackage{graphicx}
%\usepackage{Here}
\usepackage[version=4]{mhchem}
\usepackage{url}
\usepackage{setspace}
\usepackage[left=2.5cm,right=2.5cm,top=2.5cm,bottom=2cm]{geometry}
[biblography=totocnumbered]
\usepackage{fancyhdr}
\usepackage{scrextend}
\usepackage{hyperref}
\pagenumbering{gobble}

\makeatletter
\newcommand\bigcdot{\mathpalette\bigcdot@{.5}}
\newcommand\bigcdot@[2]{\mathbin{\vcenter{\hbox{\scalebox{#2}{$\m@th#1\bullet$}}}}}
\makeatother

\makeatletter
%\renewcommand*\bib@heading{%
%  \subsection*{}%
%  \@mkboth{\refname}{\refname}}
%\makeatother
\numberwithin{equation}{section}
\numberwithin{figure}{section}

\renewcommand{\labelitemii}{\labelitemfont$\vartriangleright$}
\begin{document}\\
\begin{addmargin}[25pt]{0pt}
An einer Rissspitze kommt es zu einer lokalen Spannungsüberhöhung, das bedeutet, dass dort die Spannung $\sigma_m$ deutlich höher ist als die nominal anliegende Spannung $\sigma_0$:
\begin{equation}\label{eq:Spannung_Rissspitze}
\sigma_m = 2\sigma_0 \left( \frac{a}{\rho_t}\right)^\frac{1}{2}
\end{equation}
wobei $a$ die halbe Länge des inneren, elliptischen Risses ist und $\rho_t$ der Krümmungsradius an der Riss-Spitze. Der Faktor $\frac{a}{\rho_t}$ kann sehr hohe Werte annehmen für einen langen dünnen Riss.\\
\end{addmargin}

\end{document}