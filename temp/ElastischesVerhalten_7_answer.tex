\documentclass[11pt,a4paper]{article}
\usepackage[utf8]{inputenc}
\usepackage[german]{babel}
\usepackage{amsmath}
\usepackage{amsfonts}
\usepackage{subfig}
\usepackage{amssymb}
\usepackage{siunitx,physics}
\usepackage{mathtools}
\usepackage{graphicx}
%\usepackage{Here}
\usepackage[version=4]{mhchem}
\usepackage{url}
\usepackage{setspace}
\usepackage[left=2.5cm,right=2.5cm,top=2.5cm,bottom=2cm]{geometry}
[biblography=totocnumbered]
\usepackage{fancyhdr}
\usepackage{scrextend}
\usepackage{hyperref}
\pagenumbering{gobble}

\makeatletter
\newcommand\bigcdot{\mathpalette\bigcdot@{.5}}
\newcommand\bigcdot@[2]{\mathbin{\vcenter{\hbox{\scalebox{#2}{$\m@th#1\bullet$}}}}}
\makeatother

\makeatletter
%\renewcommand*\bib@heading{%
%  \subsection*{}%
%  \@mkboth{\refname}{\refname}}
%\makeatother
\numberwithin{equation}{section}
\numberwithin{figure}{section}

\renewcommand{\labelitemii}{\labelitemfont$\vartriangleright$}
\begin{document}\\
\begin{addmargin}[25pt]{0pt}
Bei Elastomeren dominiert der entropische Teil, da die Ketten untereinander nicht sehr stark wechselwirken und somit keine Energie speichern können. Durch das Dehnen werden allerdings die Ketten gestreckt wodurch es weniger Vernetzungspunkte gibt, dadurch hat das System eine höhere Ordnung. Elastomere heizen sich bei Streckung auf und kühlen sich bei Stauchung ab. Der Elastizitätsmodul ist:
\begin{equation}\label{eq:E_Modul_Elastomere}
    E = 3nRT = \frac{3\rho RT}{M_e}
\end{equation}
Dabei ist $M_e$ die Molmasse der Ketten zwischen den Vernetzungspunkten, $\rho$ die Massendichte, $R$ die universelle Gaskonstante  und $T$ die Temperatur.
\end{addmargin} 

\end{document}